\documentclass[aspectratio=43]{beamer}

\mode<presentation> {

\usetheme{m}

\setbeamertemplate{footline} % To remove the footer line in all slides uncomment this line
%\setbeamertemplate{footline}[page number] % To replace the footer line in all slides with a simple slide count uncomment this line

%\setbeamertemplate{navigation symbols}{} % To remove the navigation symbols from the bottom of all slides uncomment this line
}

\usepackage[utf8x]{inputenc}
\usepackage{amsmath}
\usepackage[colorinlistoftodos]{todonotes}
\usepackage{amsfonts,amsmath,amssymb,amsthm,mathtools}
\usepackage{color}
\usepackage{multicol}
\usepackage{graphicx} % Allows including images
\usepackage{booktabs} % Allows the use of \toprule, \midrule and \bottomrule in tables
\usepackage{adjustbox} % for \adjincludegraphics

%----------------------------------------------------------------------------------------
%	TITLE PAGE
%----------------------------------------------------------------------------------------

\title[Short title]{Adapter Pattern}

\author{Marco Ferraioli, Dario Tecchia}
\institute[UNISA]
{
Università degli Studi di Salerno\\
marcoferraioli@live.com; dariotecchia@gmail.com\\
marcoferraioli.com
}
\date{02/11/2015} % can be changed to a custom date

\begin{document}

\begin{frame}
	\titlepage % Print the title page as the first slide
\end{frame}

%------------------------------------------------

\begin{frame}
    \textbf{Keynote}\pause
	\begin{itemize}
		\item{Introduzione Pattern Strutturali}\pause
		\item Introduzione all'Adapter Pattern\pause
		\begin{itemize}
			\item Struttura dell'Adapter Pattern\pause
			\item Tipologie di Adapter Pattern\pause
		\end{itemize}
		\item Esempi\pause
		\item Conclusioni
	\end{itemize}
\end{frame}

%----------------------------------------------------------------------------------------
%	PRESENTATION SLIDES
%----------------------------------------------------------------------------------------

\section{Introduzione Pattern Strutturali}

\begin{frame}
	\frametitle{ Pattern Strutturali}
	\textbf{Introduzione Pattern Strutturali}\pause
	\begin{itemize}
		\item I pattern strutturali consentono di riutilizzare degli oggetti esistenti fornendo agli utilizzatori un'interfaccia più adatta alle loro esigenze\pause
		\item Possono essere basati su classi o oggetti 
	\end{itemize}
\end{frame}

\begin{frame}
	\frametitle{ Pattern Strutturali}
	\textbf{Introduzione Pattern Strutturali}\pause
	\begin{itemize}
		\item Design Pattern basati su classi:\pause
		\begin{itemize}
			\item utilizzano l’ereditarietà per generare classi che combinano le proprietà di classi base\pause 
		\end{itemize}
		\item Design Pattern basati su oggetti:\pause
		\begin{itemize}
			\item ci permettono di comporre oggetti per realizzare nuove funzionalità\pause
			\item da flessibilità alla composizione che viene effettuata a run-time, cosa impossibile da realizzare con le classi 
		\end{itemize}
	\end{itemize}
\end{frame}

\section{Introduzione all'Adapter Pattern}

\begin{frame}
	\frametitle{ Adapter Pattern}
	\textbf{Introduzione all'Adapter Pattern}\pause
	\begin{itemize}
		\item L'adapter Pattern, conosciuto anche come Wrapper, si pone come soluzione astratta al problema dell'interoperabilità\\ tra interfacce differenti\pause
		\item Il problema si presenta ogni qual volta in un progetto software si devono utilizzare sistemi di supporto (es. librerie) la cui interfaccia non è perfettamente compatibile con del codice precedentemente scritto
	\end{itemize}
\end{frame}

\begin{frame}
	\centerline{Esempi di Adapter nella vita reale}
\end{frame}

\begin{frame}
	\frametitle{ Adapter Pattern}
	\begin{figure}[h!]
		\adjincludegraphics[width=0.8\linewidth,valign=t]{img/ada4.jpg}
	\end{figure}
\end{frame}

\begin{frame}
	\frametitle{ Adapter Pattern}
	\begin{figure}[h!]
		\adjincludegraphics[width=1\linewidth,valign=t]{img/ada3.jpg}
	\end{figure}
	%\text{\centerline{Adatta l'input all'output}}
\end{frame}

\begin{frame}
	\frametitle{ Adapter Pattern}
	\begin{figure}[h!]
		\adjincludegraphics[width=1\linewidth,valign=t]{img/ada2.jpg}
	\end{figure}
\end{frame}

\section{Struttura dell'Adapter Pattern}

\begin{frame}
	\frametitle{ Adapter Pattern}
	\textbf{Struttura dell'Adapter Pattern: Composizione}\pause
	\begin{itemize}
		\item \textbf{Adaptee:} definisce l’interfaccia di un diverso dominio applicativo da dover adattare per l’invocazione da parte del Client\pause
		\item \textbf{Adapter:} definisce l’interfaccia compatibile con il Target che maschera l’invocazione dell’Adaptee\pause
		\item \textbf{Target:} definisce l’interfaccia specifica del dominio applicativo utilizzata dal Client\pause
		\item \textbf{Client:} colui che effettua l’invocazione all’operazione di interesse
	\end{itemize}
\end{frame}

\section{Tipologie di Adapter Pattern}

\begin{frame}
	\frametitle{ Adapter Pattern}
	\centerline{Class Adapter}
	\begin{figure}[h!]
		\adjincludegraphics[width=1\linewidth,valign=t,scale=0.7]{img/ClassAdapter.png}
	\end{figure}
\end{frame}

\begin{frame}
	\frametitle{ Adapter Pattern}
	\textbf{Class Adapter}\pause
	\begin{itemize}
		\item  Prevede un rapporto di ereditarietà tra Adapter e Adaptee (Adapter specializza Adaptee)\pause
		\item Non è possibile creare un Adapter che specializzi più Adaptee\pause
		\item Se esiste una gerarchia di Adaptee occorre creare una gereachia di Adapter
	\end{itemize}
\end{frame}

\begin{frame}
	\frametitle{ Adapter Pattern}
	\centerline{Object Adapter}
	\begin{figure}[h!]
		\adjincludegraphics[width=1\linewidth,valign=t,scale=0.7]{img/ObjectAdapter.png}
	\end{figure}
\end{frame}

\begin{frame}
	\frametitle{ Adapter Pattern}
	\textbf{Object Adapter}\pause
	\begin{itemize}
		\item  Prevede un rapporto di associazione tra Adapter e Adaptee (Adapter instanzia Adaptee)\pause
		\item È possible avere un Adapter associato con più Adaptee
	\end{itemize}
\end{frame}

\section{Esempi}

\section{Conclusioni}

\begin{frame}
	\frametitle{ Adapter Pattern}
	\textbf{Design Pattern Collegati}\pause
	\begin{itemize}
		\item Bridge Pattern\pause
		\begin{itemize}
			\item Ha una struttura simile a quella dell’Object Adapter, ma il Bridge ha scopi differenti: separa l’interfaccia dalla propria implementazione\pause
		\end{itemize}
		\item Decorator\pause
		\begin{itemize}
			\item Il decorator è simile all'Adapter Pattern, migliora gli oggetti senza cambiarne l'interfaccia, quindi la sua implementazione risulta più trasparente \pause
		\end{itemize}
		\item Proxy\pause
		\begin{itemize}
			\item Definisce un rappresentante di un altro oggetto e non cambia la sua interfaccia
		\end{itemize}
	\end{itemize}
\end{frame}


%------------------------------------------------

\begin{frame}
    \centerline{Materiale:}
    \centerline{https://github.com/paranoiasystem/SeminarioTPA}
\end{frame}

\begin{frame}
	\frametitle{ Adapter Pattern}
	\centerline{Materiale sotto licenza GNU GPL}\pause
	\begin{figure}[h!]
		\adjincludegraphics[width=1\linewidth,valign=t,scale=0.8]{img/stallman.jpg}
	\end{figure}
\end{frame}

\begin{frame}
    \centerline{Fine, Domande?}
\end{frame}

%------------------------------------------------

\end{document} 
\grid
